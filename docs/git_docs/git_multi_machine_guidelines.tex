\documentclass[11pt,letterpaper]{article}
\usepackage[margin=1in]{geometry}
\usepackage{fancyhdr}
\usepackage{listings}
\usepackage{xcolor}
\usepackage{hyperref}
\usepackage{amsmath}
\usepackage{amssymb}
\usepackage{enumitem}

% Code formatting
\lstset{
    basicstyle=\ttfamily\small,
    backgroundcolor=\color{gray!10},
    frame=single,
    breaklines=true,
    showstringspaces=false,
    columns=flexible,
    commentstyle=\color{green!50!black},
    keywordstyle=\color{blue},
    stringstyle=\color{red},
}

% Header/Footer
\pagestyle{fancy}
\fancyhf{}
\fancyhead[L]{FractalAnalyzer Development}
\fancyhead[R]{Multi-Machine Git Guidelines}
\fancyfoot[C]{\thepage}

\title{\textbf{Multi-Machine Git Branching Guidelines\\for FractalAnalyzer Development}}
\author{Rod Douglass}
\date{\today}

\begin{document}

\maketitle

\section{Overview}

This document provides comprehensive guidelines for managing FractalAnalyzer development across multiple computing platforms (laptop, desktop, HPC systems) using Git branching strategies. The goal is to maintain code synchronization while avoiding conflicts and preserving development history.

\section{Development Environment}

\subsection{Current Setup}
\begin{itemize}
    \item \textbf{Main Repository}: \texttt{github.com/rwdlnk/Fractal\_Analyzer}
    \item \textbf{Package Structure}: Reorganized v2.0.0 with two-tier analysis system
    \item \textbf{Computing Platforms}: 
    \begin{itemize}
        \item Laptop (primary development)
        \item Desktop (computational analysis)
        \item Future: HPC systems for large-scale simulations
    \end{itemize}
\end{itemize}

\subsection{Package Organization}
\begin{lstlisting}[language=bash]
FractalAnalyzer/
├── fractal_analyzer/
│   ├── core/           # Foundation components
│   ├── analysis/       # Tier 1: General RT analysis
│   ├── validation/     # Tier 2: Dalziel validation
│   ├── utils/          # Supporting utilities
│   └── legacy/         # Archived code
├── papers/             # Research publications
├── scripts/            # Command-line tools
└── tests/              # Test suite
\end{lstlisting}

\section{Branching Strategies}

\subsection{Strategy 1: Shared Feature Branch (High Coordination)}

\textbf{Use Case}: When you want to see changes from other machines immediately.

\textbf{Workflow}:
\begin{lstlisting}[language=bash]
# Machine A - Start new feature
git checkout main
git pull origin main
git checkout -b feature/enhanced-temporal-analysis

# Do work, commit frequently
git add -A
git commit -m "Add temporal evolution improvements"
git push origin feature/enhanced-temporal-analysis

# Machine B - Continue same feature
git checkout feature/enhanced-temporal-analysis
git pull origin feature/enhanced-temporal-analysis
# Work continues seamlessly with Machine A's changes
\end{lstlisting}

\textbf{Critical Requirements}:
\begin{itemize}[label=\textbullet]
    \item \textcolor{red}{\textbf{ALWAYS}} \texttt{git pull} before starting work
    \item \textcolor{red}{\textbf{ALWAYS}} \texttt{git push} when stopping work
    \item \textcolor{red}{\textbf{NEVER}} leave uncommitted changes when switching machines
    \item \textcolor{red}{\textbf{COMMIT}} frequently to minimize conflicts
\end{itemize}

\subsection{Strategy 2: Machine-Specific Branches (Clean Separation)}

\textbf{Use Case}: When you want complete independence between machines.

\textbf{Workflow}:
\begin{lstlisting}[language=bash]
# Laptop development
git checkout -b feature/enhanced-analyzer-laptop

# Desktop development  
git checkout -b feature/enhanced-analyzer-desktop

# When ready to integrate
git checkout main
git merge feature/enhanced-analyzer-laptop
git merge feature/enhanced-analyzer-desktop
git push origin main

# Clean up branches
git branch -d feature/enhanced-analyzer-laptop
git branch -d feature/enhanced-analyzer-desktop
\end{lstlisting}

\textbf{Advantages}:
\begin{itemize}[label=\checkmark]
    \item No coordination required
    \item No risk of conflicts during development
    \item Clean, isolated development environments
\end{itemize}

\textbf{Disadvantages}:
\begin{itemize}[label=\textbullet]
    \item Cannot see other machine's progress until merge
    \item Potential for duplicate work
    \item More complex integration at end
\end{itemize}

\subsection{Strategy 3: Hybrid Approach (Recommended)}

\textbf{Use Case}: Best of both worlds - coordination with safety.

\textbf{Workflow}:
\begin{lstlisting}[language=bash]
# Start with shared feature branch
git checkout -b feature/dalziel-validation-improvements

# Create machine-specific sub-branches
# On laptop:
git checkout -b feature/dalziel-validation-laptop

# On desktop:
git checkout -b feature/dalziel-validation-desktop

# Periodic synchronization (e.g., daily)
# Laptop pushes to shared branch:
git checkout feature/dalziel-validation-laptop
git commit -am "Laptop: Add power spectrum analysis"
git checkout feature/dalziel-validation-improvements
git merge feature/dalziel-validation-laptop
git push origin feature/dalziel-validation-improvements

# Desktop pulls integrated work:
git checkout feature/dalziel-validation-improvements
git pull origin feature/dalziel-validation-improvements
git checkout feature/dalziel-validation-desktop
git merge feature/dalziel-validation-improvements
# Desktop now has laptop's work + can continue independently
\end{lstlisting}

\section{Branch Naming Conventions}

\subsection{Feature Development}
\begin{lstlisting}[language=bash]
feature/enhanced-analyzer-improvements
feature/dalziel-validation-updates
feature/multifractal-analysis
feature/performance-optimization
\end{lstlisting}

\subsection{Bug Fixes}
\begin{lstlisting}[language=bash]
fix/import-error-core-modules
fix/vtk-reader-memory-leak
fix/convergence-analysis-edge-case
\end{lstlisting}

\subsection{Research/Papers}
\begin{lstlisting}[language=bash]
paper/validation-results-2025
research/temporal-evolution-study
paper/jfm-submission-rev1
\end{lstlisting}

\subsection{Machine-Specific (when using Strategy 2 or 3)}
\begin{lstlisting}[language=bash]
feature/enhanced-analyzer-laptop
feature/enhanced-analyzer-desktop
feature/enhanced-analyzer-hpc
\end{lstlisting}

\section{Best Practices}

\subsection{Before Starting Work (Any Strategy)}
\begin{lstlisting}[language=bash]
# Check current status
git status

# Ensure you're on the right branch
git branch

# Get latest changes
git pull origin <branch-name>

# Verify package installation is current
python -c "import fractal_analyzer; print(f'v{fractal_analyzer.__version__}')"
\end{lstlisting}

\subsection{During Development}
\begin{itemize}
    \item \textbf{Commit Early, Commit Often}: Small, logical commits with clear messages
    \item \textbf{Test Imports}: Ensure package structure remains intact
    \item \textbf{Document Changes}: Update docstrings and comments
    \item \textbf{Run Tests}: Verify functionality before pushing
\end{itemize}

\subsection{Ending Work Session}
\begin{lstlisting}[language=bash]
# Commit all changes
git add -A
git commit -m "Descriptive commit message"

# Push to remote (Strategy 1 and 3)
git push origin <branch-name>

# Optional: Tag important milestones
git tag -a v2.1.0 -m "Enhanced temporal analysis complete"
git push origin v2.1.0
\end{lstlisting}

\section{Conflict Resolution}

\subsection{When Conflicts Occur}
\begin{lstlisting}[language=bash]
# If merge conflict during pull
git status                    # See conflicted files
# Edit files to resolve conflicts (remove <<<, ===, >>> markers)
git add <resolved-files>
git commit -m "Resolve merge conflicts"

# If you need to abort and start over
git merge --abort
git reset --hard HEAD
\end{lstlisting}

\subsection{Prevention Strategies}
\begin{itemize}
    \item Work on different modules when possible
    \item Communicate about simultaneous work on same files
    \item Use Strategy 2 or 3 for complex, overlapping changes
    \item Keep main branch stable and tested
\end{itemize}

\section{Integration Workflow}

\subsection{Merging Feature to Main}
\begin{lstlisting}[language=bash]
# Ensure feature is complete and tested
git checkout feature/my-feature
python fractal_analyzer/analysis/enhanced_analyzer.py --help  # Test
python -c "import fractal_analyzer; print('Package works')"    # Test

# Update main and merge
git checkout main
git pull origin main
git merge feature/my-feature

# Test integrated system
pip install -e .  # Reinstall package
python -c "import fractal_analyzer; print(f'v{fractal_analyzer.__version__}')"

# Push and clean up
git push origin main
git branch -d feature/my-feature
git push origin --delete feature/my-feature  # Delete remote branch
\end{lstlisting}

\section{Recommended Strategy by Use Case}

\begin{center}
\begin{tabular}{|l|c|c|c|}
\hline
\textbf{Scenario} & \textbf{Strategy 1} & \textbf{Strategy 2} & \textbf{Strategy 3} \\
\hline
Small bug fixes & \checkmark & & \\
\hline
Single-day features & \checkmark & & \\
\hline
Multi-day features & & & \checkmark \\
\hline
Independent research & & \checkmark & \\
\hline
Collaborative features & \checkmark & & \checkmark \\
\hline
Experimental changes & & \checkmark & \\
\hline
Paper deadlines & \checkmark & & \\
\hline
\end{tabular}
\end{center}

\section{Emergency Procedures}

\subsection{If You Forget to Pull Before Working}
\begin{lstlisting}[language=bash]
# Stash your changes
git stash

# Pull latest changes
git pull origin <branch-name>

# Apply your stashed changes
git stash pop

# Resolve any conflicts, then commit
\end{lstlisting}

\subsection{If You Need to Sync Immediately}
\begin{lstlisting}[language=bash]
# Quick sync without full merge
git fetch origin
git log HEAD..origin/<branch-name> --oneline  # See what's new
git pull origin <branch-name>
\end{lstlisting}

\section{Conclusion}

The hybrid approach (Strategy 3) is recommended for most FractalAnalyzer development scenarios, as it provides both coordination benefits and development safety. Always prioritize keeping the main branch stable and tested, as it serves as the foundation for your Rayleigh-Taylor instability research.

Remember: \textit{When in doubt, create a new branch!} Branches are cheap, but lost work is expensive.

\end{document}
