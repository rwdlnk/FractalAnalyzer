\documentclass[12pt,a4paper]{article}
\usepackage{amsmath,amssymb,amsfonts}
\usepackage{graphicx}
\usepackage{hyperref}
\usepackage{physics}
\usepackage[numbers]{natbib}
\usepackage{float}
\usepackage{xcolor}

\title{Linear Stability Analysis of the Rayleigh-Taylor Instability in Two Immiscible, Inviscid Fluid Layers}
\author{Claude (Anthropic)}
\date{\today}

\begin{document}

\maketitle

\begin{abstract}
    This article presents a detailed linear stability analysis of the Rayleigh-Taylor instability occurring at the interface between two immiscible, inviscid fluid layers of different densities. The system consists of two fluids separated by a horizontal interface, with rigid boundaries at the top and bottom. We derive the governing equations from first principles, solve for the velocity field in both fluid regions, and establish the dispersion relation that determines the stability of the system. The effects of surface tension and finite fluid depth are incorporated into the analysis. We find that the system is unstable when the heavier fluid overlies the lighter fluid, unless sufficiently strong surface tension is present. The complete mathematical description of the velocity field is provided, which can serve as a foundation for further investigations into the nonlinear development and fractal characteristics of the evolving interface.
\end{abstract}

\section{Introduction}
The Rayleigh-Taylor instability (RTI) is a fundamental fluid instability that occurs when a denser fluid is supported by a lighter fluid against gravity \citep{rayleigh1882, taylor1950}. This instability plays a crucial role in numerous natural and engineered systems, from supernovae explosions to inertial confinement fusion. The interface between the two fluids develops characteristic structures that evolve into increasingly complex patterns, eventually exhibiting fractal-like properties \citep{sharp1984, youngs1991}.

Understanding the linear stability of the RTI is the first step in analyzing its more complex nonlinear development. In this article, we present a comprehensive linear stability analysis of two immiscible, inviscid fluid layers separated by a horizontal interface, incorporating the effects of surface tension and finite fluid depth.

The results of this analysis provide the mathematical foundation for studying the evolution of the interface, which can later be analyzed using fractal dimension measurements such as the box-counting method. This approach is particularly useful for quantifying the multi-scale structures that develop during the nonlinear phase of the instability.

\section{Problem Formulation}
\subsection{Geometric Configuration}
We consider two immiscible, inviscid fluids each having a constant, but different density occupying a space between two horizontal planes. The interface between the fluids is initially parallel to the bounding planes forming a horizontal plane. We establish a coordinate system such that the $z$-axis is vertical and the interface lies in the $x$-$y$ plane at $z = 0$. The top boundary is at $z = H_t$ and the bottom at $z = -H_b$, with the boundaries being $H_b + H_t$ apart.

Constant gravitational acceleration acts in the $-z$ direction with magnitude $g$. The bottom fluid (fluid 1) has density $\rho_1$, and the top fluid (fluid 2) has density $\rho_2$. Initially, the fluid is at rest.

\subsection{Governing Equations}
For small-amplitude perturbations, the linearized equations for inviscid, incompressible flow are:
\begin{align}
\nabla \cdot \mathbf{v} &= 0 \\
\rho \frac{\partial \mathbf{v}}{\partial t} &= -\nabla p + \rho \mathbf{g}
\end{align}

Taking the curl of the momentum equation eliminates the pressure gradient:
\begin{equation}
\frac{\partial}{\partial t}(\nabla \times \mathbf{v}) = 0
\end{equation}

This implies that in initially irrotational flow, the flow remains irrotational:
\begin{equation}
\nabla \times \mathbf{v} = 0
\end{equation}

For irrotational flow, we can introduce a velocity potential $\phi$ such that $\mathbf{v} = \nabla \phi$. The incompressibility condition then becomes the Laplace equation:
\begin{equation}
\nabla^2 \phi = 0
\end{equation}

For normal mode analysis, we consider perturbations of the form:
\begin{equation}
\phi(x,y,z,t) = \Phi(z)e^{i(k_x x + k_y y - \omega t)}
\end{equation}

Substituting into the Laplace equation:
\begin{equation}
\frac{d^2\Phi}{dz^2} - k^2\Phi = 0
\end{equation}

where $k^2 = k_x^2 + k_y^2$.

The vertical velocity component is $w = \partial\phi/\partial z$, which satisfies:
\begin{equation}
\frac{d^2 w}{dz^2} - k^2 w = 0, \quad -H_b < z < H_t
\end{equation}

\subsection{Boundary Conditions}
The boundary conditions for this problem are:

1. At the rigid boundaries, the vertical velocity must vanish:
\begin{align}
w_1(-H_b) &= 0 \\
w_2(H_t) &= 0
\end{align}

2. At the interface, the vertical velocity must be continuous:
\begin{equation}
w_1(0) = w_2(0) = w_0
\end{equation}

3. The pressure balance at the interface, accounting for surface tension $T$, yields:
\begin{equation}
\Delta_0(\rho D(w)) = -\frac{k^2}{\omega^2}[g(\rho_2 - \rho_1) - k^2T]w_0
\end{equation}

where $D() = d/dz$ and $\Delta_0(f) = [f(z+0) - f(z-0)]_{z=0}$.

\section{Solution Methodology}
\subsection{General Solution for Vertical Velocity}
The general solution to the differential equation $\frac{d^2 w}{dz^2} - k^2 w = 0$ is:
\begin{equation}
w(z) = Ae^{kz} + Be^{-kz}
\end{equation}

Applying this form to each fluid region:

For fluid 1 ($-H_b < z < 0$):
\begin{equation}
w_1(z) = A_1e^{kz} + B_1e^{-kz}
\end{equation}

For fluid 2 ($0 < z < H_t$):
\begin{equation}
w_2(z) = A_2e^{kz} + B_2e^{-kz}
\end{equation}

\subsection{Application of Boundary Conditions}
Applying the boundary condition $w_1(-H_b) = 0$:
\begin{equation}
A_1e^{-kH_b} + B_1e^{kH_b} = 0
\end{equation}

This gives:
\begin{equation}
A_1 = -B_1e^{2kH_b}
\end{equation}

Substituting back:
\begin{equation}
w_1(z) = B_1(e^{-kz} - e^{2kH_b+kz})
\end{equation}

Similarly, applying $w_2(H_t) = 0$:
\begin{equation}
A_2e^{kH_t} + B_2e^{-kH_t} = 0
\end{equation}

This gives:
\begin{equation}
A_2 = -B_2e^{-2kH_t}
\end{equation}

Substituting back:
\begin{equation}
w_2(z) = B_2(e^{-kz} - e^{-2kH_t+kz})
\end{equation}

\subsection{Interface Conditions and Dispersion Relation}
From the continuity condition $w_1(0) = w_2(0) = w_0$:
\begin{align}
B_1(1 - e^{2kH_b}) &= w_0 \\
B_2(1 - e^{-2kH_t}) &= w_0
\end{align}

This gives:
\begin{align}
B_1 &= \frac{w_0}{1 - e^{2kH_b}} \\
B_2 &= \frac{w_0}{1 - e^{-2kH_t}}
\end{align}

Computing the derivatives at the interface:
\begin{align}
\left.\frac{dw_1}{dz}\right|_{z=0} &= -kB_1(1 + e^{2kH_b}) \\
\left.\frac{dw_2}{dz}\right|_{z=0} &= -kB_2(1 + e^{-2kH_t})
\end{align}

Applying the pressure balance condition:
\begin{equation}
\rho_2\left.\frac{dw_2}{dz}\right|_{z=0} - \rho_1\left.\frac{dw_1}{dz}\right|_{z=0} = -\frac{k^2}{\omega^2}[g(\rho_2 - \rho_1) - k^2T]w_0
\end{equation}

Substituting the derivatives and rearranging:
\begin{equation}
-k\rho_2\frac{1 + e^{-2kH_t}}{1 - e^{-2kH_t}}w_0 + k\rho_1\frac{1 + e^{2kH_b}}{1 - e^{2kH_b}}w_0 = -\frac{k^2}{\omega^2}[g(\rho_2 - \rho_1) - k^2T]w_0
\end{equation}

Simplifying with the hyperbolic cotangent function:
\begin{equation}
\rho_2\coth(kH_t) + \rho_1\coth(kH_b) = \frac{k}{\omega^2}[g(\rho_2 - \rho_1) - k^2T]
\end{equation}

Solving for $\omega^2$:
\begin{equation}
\omega^2 = \frac{k[g(\rho_2 - \rho_1) - k^2T]}{\rho_1\coth(kH_b) + \rho_2\coth(kH_t)}
\end{equation}

\section{Complete Velocity Field Solution}
\subsection{Vertical Velocity Components}
The vertical velocity components in each fluid region are:

For fluid 1 ($-H_b < z < 0$):
\begin{equation}
w_1(z) = \frac{w_0}{1 - e^{2kH_b}}(e^{-kz} - e^{2kH_b+kz})
\end{equation}

For fluid 2 ($0 < z < H_t$):
\begin{equation}
w_2(z) = \frac{w_0}{1 - e^{-2kH_t}}(e^{-kz} - e^{-2kH_t+kz})
\end{equation}

\subsection{Horizontal Velocity Components}
From the incompressibility condition and irrotationality:
\begin{align}
\frac{\partial u}{\partial x} + \frac{\partial v}{\partial y} + \frac{\partial w}{\partial z} &= 0 \\
\frac{\partial u}{\partial y} - \frac{\partial v}{\partial x} &= 0 \\
\frac{\partial w}{\partial x} - \frac{\partial u}{\partial z} &= 0 \\
\frac{\partial w}{\partial y} - \frac{\partial v}{\partial z} &= 0
\end{align}

For perturbations of the form $e^{i(k_x x + k_y y - \omega t)}$:
\begin{align}
ik_xu + ik_yv + \frac{dw}{dz} &= 0 \\
ik_yu - ik_xv &= 0 \\
ik_xw - \frac{du}{dz} &= 0 \\
ik_yw - \frac{dv}{dz} &= 0
\end{align}

From these relations:
\begin{align}
u &= -i\frac{k_x}{k^2}\frac{dw}{dz} \\
v &= -i\frac{k_y}{k^2}\frac{dw}{dz}
\end{align}

Applying to each fluid region:

For fluid 1:
\begin{align}
u_1(z) &= i\frac{k_x}{k}\frac{w_0}{1 - e^{2kH_b}}(e^{-kz} + e^{2kH_b+kz}) \\
v_1(z) &= i\frac{k_y}{k}\frac{w_0}{1 - e^{2kH_b}}(e^{-kz} + e^{2kH_b+kz})
\end{align}

For fluid 2:
\begin{align}
u_2(z) &= i\frac{k_x}{k}\frac{w_0}{1 - e^{-2kH_t}}(e^{-kz} + e^{-2kH_t+kz}) \\
v_2(z) &= i\frac{k_y}{k}\frac{w_0}{1 - e^{-2kH_t}}(e^{-kz} + e^{-2kH_t+kz})
\end{align}

\subsection{Time-Dependent Solution}
The complete time-dependent solution is:

For fluid 1 ($-H_b < z < 0$):
\begin{align}
w_1(x,y,z,t) &= \frac{w_0}{1 - e^{2kH_b}}(e^{-kz} - e^{2kH_b+kz})e^{i(k_xx + k_yy - \omega t)} \\
u_1(x,y,z,t) &= i\frac{k_x}{k}\frac{w_0}{1 - e^{2kH_b}}(e^{-kz} + e^{2kH_b+kz})e^{i(k_xx + k_yy - \omega t)} \\
v_1(x,y,z,t) &= i\frac{k_y}{k}\frac{w_0}{1 - e^{2kH_b}}(e^{-kz} + e^{2kH_b+kz})e^{i(k_xx + k_yy - \omega t)}
\end{align}

For fluid 2 ($0 < z < H_t$):
\begin{align}
w_2(x,y,z,t) &= \frac{w_0}{1 - e^{-2kH_t}}(e^{-kz} - e^{-2kH_t+kz})e^{i(k_xx + k_yy - \omega t)} \\
u_2(x,y,z,t) &= i\frac{k_x}{k}\frac{w_0}{1 - e^{-2kH_t}}(e^{-kz} + e^{-2kH_t+kz})e^{i(k_xx + k_yy - \omega t)} \\
v_2(x,y,z,t) &= i\frac{k_y}{k}\frac{w_0}{1 - e^{-2kH_t}}(e^{-kz} + e^{-2kH_t+kz})e^{i(k_xx + k_yy - \omega t)}
\end{align}

\section{Stability Analysis}
\subsection{Stability Criterion}
The stability of the system is determined by the sign of $\omega^2$ in the dispersion relation:
\begin{equation}
\omega^2 = \frac{k[g(\rho_2 - \rho_1) - k^2T]}{\rho_1\coth(kH_b) + \rho_2\coth(kH_t)}
\end{equation}

Since the denominator $\rho_1\coth(kH_b) + \rho_2\coth(kH_t)$ is always positive (as both $\coth(kH_b)$ and $\coth(kH_t)$ are positive for positive arguments), the sign of $\omega^2$ is determined by the numerator $g(\rho_2 - \rho_1) - k^2T$.

The system is stable ($\omega^2 > 0$) when:
\begin{equation}
g(\rho_2 - \rho_1) - k^2T < 0
\end{equation}

This occurs in two scenarios:
\begin{enumerate}
    \item When $\rho_2 < \rho_1$ (heavier fluid on bottom), the term $g(\rho_2 - \rho_1)$ is negative, ensuring stability regardless of surface tension.
    \item When $\rho_2 > \rho_1$ (heavier fluid on top), stability requires that surface tension be sufficiently large: $k^2T > g(\rho_2 - \rho_1)$.
\end{enumerate}

The system is unstable ($\omega^2 < 0$) when:
\begin{equation}
g(\rho_2 - \rho_1) - k^2T > 0 \quad \text{and} \quad \rho_2 > \rho_1
\end{equation}

In this case, perturbations grow exponentially with time according to $e^{\gamma t}$, where $\gamma = \sqrt{-\omega^2}$ is the growth rate.

\subsection{Cutoff Wavenumber}
When $\rho_2 > \rho_1$, there exists a cutoff wavenumber $k_c$ above which the system is stable due to surface tension:
\begin{equation}
k_c = \sqrt{\frac{g(\rho_2 - \rho_1)}{T}}
\end{equation}

Perturbations with $k > k_c$ are stable, while those with $k < k_c$ are unstable. This indicates that surface tension stabilizes short-wavelength perturbations, while long-wavelength perturbations can still grow.

\subsection{Most Unstable Mode}
The growth rate as a function of wavenumber is:
\begin{equation}
\gamma(k) = \sqrt{\frac{k[g(\rho_2 - \rho_1) - k^2T]}{\rho_1\coth(kH_b) + \rho_2\coth(kH_t)}}
\end{equation}

To find the most unstable mode, we need to determine where $\frac{d\gamma}{dk} = 0$. Let's define $f(k) = \gamma^2(k)$ to simplify:
\begin{equation}
f(k) = \frac{k[g(\rho_2 - \rho_1) - k^2T]}{\rho_1\coth(kH_b) + \rho_2\coth(kH_t)}
\end{equation}

Taking the derivative with respect to $k$:
\begin{align}
\frac{df}{dk} &= \frac{[g(\rho_2 - \rho_1) - k^2T - 2k^2T][\rho_1\coth(kH_b) + \rho_2\coth(kH_t)] - k[g(\rho_2 - \rho_1) - k^2T]\frac{d}{dk}[\rho_1\coth(kH_b) + \rho_2\coth(kH_t)]}{[\rho_1\coth(kH_b) + \rho_2\coth(kH_t)]^2} \\
&= \frac{[g(\rho_2 - \rho_1) - 3k^2T][\rho_1\coth(kH_b) + \rho_2\coth(kH_t)] - k[g(\rho_2 - \rho_1) - k^2T]\frac{d}{dk}[\rho_1\coth(kH_b) + \rho_2\coth(kH_t)]}{[\rho_1\coth(kH_b) + \rho_2\coth(kH_t)]^2}
\end{align}

Where:
\begin{equation}
\frac{d}{dk}[\coth(kH)] = -H \cdot \text{csch}^2(kH)
\end{equation}

Thus:
\begin{equation}
\frac{d}{dk}[\rho_1\coth(kH_b) + \rho_2\coth(kH_t)] = -\rho_1 H_b \cdot \text{csch}^2(kH_b) - \rho_2 H_t \cdot \text{csch}^2(kH_t)
\end{equation}

Setting $\frac{df}{dk} = 0$ and solving for $k$ gives the most unstable mode. In the general case, this requires solving:
\begin{multline}
[g(\rho_2 - \rho_1) - 3k^2T][\rho_1\coth(kH_b) + \rho_2\coth(kH_t)] + \\
k[g(\rho_2 - \rho_1) - k^2T][\rho_1 H_b \cdot \text{csch}^2(kH_b) + \rho_2 H_t \cdot \text{csch}^2(kH_t)] = 0
\end{multline}

The derivative of the growth rate $\gamma$ with respect to $k$ is:
\begin{equation}
\frac{d\gamma}{dk} = \frac{1}{2\gamma} \cdot \frac{df}{dk}
\end{equation}

\subsubsection{Special Cases}
For specific cases, we can find analytical solutions:

\paragraph{Deep Fluid Limit:} When $kH_b, kH_t \gg 1$, we have $\coth(kH_b) \approx \coth(kH_t) \approx 1$ and $\text{csch}^2(kH_b) \approx \text{csch}^2(kH_t) \approx 0$. The equation simplifies to:
\begin{equation}
[g(\rho_2 - \rho_1) - 3k^2T](\rho_1 + \rho_2) = 0
\end{equation}

Solving for $k$:
\begin{equation}
k_{max} = \sqrt{\frac{g(\rho_2 - \rho_1)}{3T}}
\end{equation}

\paragraph{Equal Depths:} When $H_b = H_t = H$, the equation still requires numerical solution except in limiting cases.

\paragraph{Shallow Fluid Limit:} When $kH_b, kH_t \ll 1$, we have $\coth(kH) \approx \frac{1}{kH}$ and the equation becomes:
\begin{equation}
\frac{df}{dk} \approx \frac{[g(\rho_2 - \rho_1) - 3k^2T][\frac{\rho_1}{kH_b} + \frac{\rho_2}{kH_t}] - k[g(\rho_2 - \rho_1) - k^2T][-\frac{\rho_1}{k^2H_b} - \frac{\rho_2}{k^2H_t}]}{[\frac{\rho_1}{kH_b} + \frac{\rho_2}{kH_t}]^2}
\end{equation}

After simplification and solving for $k_{max}$, we find that in the shallow fluid limit, the most unstable wavenumber can be approximated as:
\begin{equation}
k_{max} \approx \sqrt{\frac{g(\rho_2 - \rho_1)}{5T}}
\end{equation}

This indicates that in shallow fluid layers, the most unstable wavelength is longer compared to the deep fluid case.

\section{Asymptotic Behavior}
\subsection{Deep Fluid Limit}
In the limit of deep fluid layers ($kH_b, kH_t \gg 1$), we have $\coth(kH_b) \approx \coth(kH_t) \approx 1$, and the dispersion relation simplifies to:
\begin{equation}
\omega^2 \approx \frac{k[g(\rho_2 - \rho_1) - k^2T]}{\rho_1 + \rho_2}
\end{equation}

This is the classical result for the Rayleigh-Taylor instability in semi-infinite fluids.

\subsection{Shallow Fluid Limit}
In the limit of shallow fluid layers ($kH_b, kH_t \ll 1$), we have $\coth(kH_b) \approx \frac{1}{kH_b}$ and $\coth(kH_t) \approx \frac{1}{kH_t}$, leading to:
\begin{equation}
\omega^2 \approx \frac{k[g(\rho_2 - \rho_1) - k^2T]}{\frac{\rho_1}{kH_b} + \frac{\rho_2}{kH_t}}
\end{equation}

Simplifying:
\begin{equation}
\omega^2 \approx \frac{k^2H_bH_t[g(\rho_2 - \rho_1) - k^2T]}{\rho_1H_t + \rho_2H_b}
\end{equation}

This indicates that in shallow layers, the growth rate is reduced, providing a stabilizing effect.

\section{Discussion and Implications for Fractal Analysis}
The linear stability analysis presented here provides the foundation for understanding the initial development of the Rayleigh-Taylor instability. However, the most interesting aspects of this instability emerge in the nonlinear regime, where the interface develops complex structures with potential fractal characteristics.

\subsection{Connection to Fractal Analysis}
As the Rayleigh-Taylor instability develops, the initially flat interface evolves into a complex structure characterized by bubbles of lighter fluid rising into the heavier fluid and spikes of heavier fluid falling into the lighter fluid. Over time, secondary instabilities such as Kelvin-Helmholtz develop along the sides of these structures, creating increasingly complex patterns.

The box-counting method can be applied to measure the fractal dimension of this evolving interface. For a curve in two-dimensional space, the box-counting dimension $D$ is defined as:
\begin{equation}
D = \lim_{\epsilon \to 0} \frac{\log N(\epsilon)}{\log(1/\epsilon)}
\end{equation}

where $N(\epsilon)$ is the number of boxes of size $\epsilon$ required to cover the curve. For the Rayleigh-Taylor interface, this dimension is expected to increase from $D = 1$ (smooth interface) to values between 1 and 2 as the instability develops.

\subsection{Expected Evolution of Fractal Dimension}
Based on the linear stability analysis, we can make some predictions about the evolution of the fractal dimension:

1. In the early stages, dominated by the linear regime, the interface will remain relatively smooth with $D \approx 1$.

2. As the most unstable modes grow, the interface will develop regular structures determined by the wavelength $\lambda_{max} = \frac{2\pi}{k_{max}}$.

3. When secondary instabilities develop, the fractal dimension will increase rapidly as smaller-scale structures emerge.

4. In the fully developed nonlinear regime, the fractal dimension may reach a characteristic value that depends on the Atwood number $A = \frac{\rho_2 - \rho_1}{\rho_2 + \rho_1}$ and possibly other parameters.

Surface tension, which stabilizes small-scale perturbations, is expected to impose a lower cutoff on the range of scales exhibiting fractal behavior, potentially leading to a multifractal structure.

\section{Conclusion}
We have presented a comprehensive linear stability analysis of the Rayleigh-Taylor instability for two immiscible, inviscid fluid layers separated by a horizontal interface. The complete velocity field solution has been derived, and the stability criterion has been established.

The key findings are:
\begin{enumerate}
    \item The system is stable when the heavier fluid is on the bottom ($\rho_2 < \rho_1$).
    \item When the heavier fluid is on top ($\rho_2 > \rho_1$), the system is unstable unless surface tension is sufficiently strong.
    \item Surface tension stabilizes short-wavelength perturbations, creating a cutoff wavenumber $k_c = \sqrt{\frac{g(\rho_2 - \rho_1)}{T}}$.
    \item Finite fluid depth provides additional stabilization compared to the semi-infinite case.
\end{enumerate}

This analysis provides the mathematical foundation for further investigations into the nonlinear development of the Rayleigh-Taylor instability and its fractal characteristics. The box-counting method can be applied to the evolving interface to quantify its increasing complexity, providing insights into the multi-scale nature of this fundamental fluid instability.

\section*{Acknowledgements}
This analysis was prepared by Claude (Anthropic) at the request of the researcher on March 29, 2025. This document may be freely used for research purposes with appropriate attribution.

\begin{thebibliography}{9}

\bibitem[Rayleigh(1882)]{rayleigh1882}
Rayleigh, L. (1882).
\newblock {Investigation of the Character of the Equilibrium of an Incompressible Heavy Fluid of Variable Density}.
\newblock {\em Proceedings of the London Mathematical Society}, 14(1):170--177.

\bibitem[Taylor(1950)]{taylor1950}
Taylor, G.~I. (1950).
\newblock {The Instability of Liquid Surfaces When Accelerated in a Direction Perpendicular to Their Planes. I}.
\newblock {\em Proceedings of the Royal Society of London A}, 201(1065):192--196.

\bibitem[Sharp(1984)]{sharp1984}
Sharp, D.~H. (1984).
\newblock {An Overview of Rayleigh-Taylor Instability}.
\newblock {\em Physica D: Nonlinear Phenomena}, 12(1-3):3--18.

\bibitem[Youngs(1991)]{youngs1991}
Youngs, D.~L. (1991).
\newblock {Three-Dimensional Numerical Simulation of Turbulent Mixing by Rayleigh-Taylor Instability}.
\newblock {\em Physics of Fluids A: Fluid Dynamics}, 3(5):1312--1320.

\bibitem[Mandelbrot(1982)]{mandelbrot1982}
Mandelbrot, B.~B. (1982).
\newblock {\em {The Fractal Geometry of Nature}}.
\newblock W.~H. Freeman and Company, San Francisco.

\bibitem[Cheng et~al.(2002)]{cheng2002}
Cheng, B., Glimm, J., and Sharp, D.~H. (2002).
\newblock {Dynamical Evolution of Rayleigh-Taylor and Richtmyer-Meshkov Mixing Fronts}.
\newblock {\em Physical Review E}, 66(3):036312.

\end{thebibliography}

\end{document}
